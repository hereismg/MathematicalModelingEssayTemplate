\thispagestyle{empty}   % 定义起始页的页眉页脚格式为 empty —— 空,也就没有页眉页脚

\begin{table}[h]
    \centering
    \begin{tabular}{|c|c|}
    \hline
    队伍编号 & MC2311870 \\ \hline
    题号   & B         \\ \hline
    \end{tabular}
\end{table}



\begin{center}
    \textbf{—————————————————————————————————}

    \textbf{\fontsize{20}{1.5}通过遗传算法和仿真模拟优化列车时刻表问题}

    \textbf{摘 要}
\end{center}





% ==================================================
%
%   摘要
%
% --------------------------------------------------

这里是摘要。这里是摘要。这里是摘要。这里是摘要。这里是摘要。这里是摘要。这里是摘要。这里是摘要。这里是摘要。这里是摘要。这里是摘要。这里是摘要。这里是摘要。这里是摘要。这里是摘要。这里是摘要。这里是摘要。这里是摘要。这里是摘要。这里是摘要。这里是摘要。

这里也是摘要。这里也是摘要。这里也是摘要。这里也是摘要。这里也是摘要。这里也是摘要。这里也是摘要。这里也是摘要。这里也是摘要。这里也是摘要。这里也是摘要。这里也是摘要。这里也是摘要。这里也是摘要。这里也是摘要。这里也是摘要。这里也是摘要。这里也是摘要。这里也是摘要。

这里还是摘要。这里还是摘要。这里还是摘要。这里还是摘要。这里还是摘要。这里还是摘要。这里还是摘要。这里还是摘要。这里还是摘要。这里还是摘要。这里还是摘要。这里还是摘要。这里还是摘要。这里还是摘要。这里还是摘要。这里还是摘要。这里还是摘要。这里还是摘要。这里还是摘要。\newline
\newline
\textbf{关键词}:关键词一 关键词二 关键词三 关键词四



% ==================================================
%
%   目录
%
% --------------------------------------------------

% 下面添加一个目录,同样定义目录页眉页脚格式为 empty —— 空
\newpage
\tableofcontents
\thispagestyle{empty}