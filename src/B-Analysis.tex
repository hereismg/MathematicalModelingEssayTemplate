% 设置页码计数器为 1 (也就是当前页面为第一页)
\setcounter{page}{1}

% ==================================================
%
%   问题重述
%
% --------------------------------------------------

\mcmSection{问题重述}

\mcmSubsection{问题背景}

这里是问题的重述。这里是问题的重述。这里是问题的重述。这里是问题的重述。这里是问题的重述。这里是问题的重述。这里是问题的重述。这里是问题的重述。这里是问题的重述。这里是问题的重述。这里是问题的重述。这里是问题的重述。这里是问题的重述。

\mcmSubsection{具体问题重述}

这里还是问题的重述。这里还是问题的重述。这里还是问题的重述。这里还是问题的重述。这里还是问题的重述。这里还是问题的重述。这里还是问题的重述。这里还是问题的重述。这里还是问题的重述。这里还是问题的重述。这里还是问题的重述。这里还是问题的重述。这里还是问题的重述。


% ==================================================
% @brief    问题分析
% ==================================================

\mcmSection{问题分析}

\mcmSubsection{分析}

这里是问题的分析。这里是问题的分析。这里是问题的分析。这里是问题的分析。这里是问题的分析。这里是问题的分析。这里是问题的分析。这里是问题的分析。这里是问题的分析。这里是问题的分析。这里是问题的分析。这里是问题的分析。这里是问题的分析。这里是问题的分析。这里是问题的分析。这里是问题的分析。这里是问题的分析。这里是问题的分析。这里是问题的分析。


% ==================================================
% @brief    模型假设
% ==================================================

\mcmSection{模型假设}

\begin{enumerate}
    \item 假设一
    \item 假设二
    \item 假设三
\end{enumerate}

% ==================================================
% @brief    模型假设
% ==================================================

\mcmSection{符号说明及名称定义}

\begin{table}[]
    % 表格居中
    \centering

    % 调整行距
    \renewcommand\arraystretch{1.5}
    
    % 放缩表格
    \scalebox{1.2}{

        \begin{tabular}{cc}
        \hline
        \textbf{\fontsize{13}{1.5}{符号}}            & \textbf{\fontsize{13}{1.5}{意义}}                       \\ \hline

        $Z_{n,l}$     & 文档$n$中第$l$个分词对应的主题编号     \\
        $\alpha$      & 控制文档-主题分布的超参数/狄利克雷分布先验参数 \\
        $\beta$       & 控制主题-词语分布的超参数/狄利克雷分布先验参数 \\
        $w_{n,l}$     & 文档$n$中第$l$个分词对应的主题编号     \\
        $\varphi_{k}$ & 主题$k$的词汇分布               \\
        $M_{diff}$    & 题目难度矩阵                   \\
        $L_{weight}$  & 题目难度值                    \\
        $L_D$         & 题目难度系数序列                 \\
        $S_{level}$   & 难度度量等级集合                 \\ \hline
        \end{tabular}
    
    }
\end{table}
